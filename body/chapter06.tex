% !TEX root = ../main.tex

% 中英标题:\chapter{中文标题}[英文标题]
\chapter{补充说明}[Complements]

\section{引言}[Introduction]

这是 \hitszthesis\ 的示例文档,基本上覆盖了模板中所有格式的设置。建议大家在使用模
板之前,除了阅读《\hitszthesis\:哈尔滨工业大学学位论文模板》\footnote{即
hitszthesis.pdf文件},本示例文档也最好能看一看。此示例文档尽量使用到所有的排版格式
,然而对于一些不在我工规范中规定的文档,理论上是由用户自由发挥,这里不给出样例
。需要另行载入的宏包和自定义命令在文件`hitszthesis.sty'中有示例,这里不列举。

\section{关于数字}[Number]

按《关于出版物上数字用法的试行规定》(1987年1月1日国家语言文字工作委员会等7个单位公布),除习惯用中文数字表示的以外,一般数字均用阿拉伯数字。
(1)公历的世纪、年代、年、月、日和时刻一律用阿拉伯数字,如20世纪,80年代,4时3刻等。年号要用四位数,如1989年,不能用89年。
(2)记数与计算(含正负整数、分数、小数、百分比、约数等)一律用阿拉伯数字,如3/4,4.5\%,10个月,500多种等。
(3)一个数值的书写形式要照顾到上下文。不是出现在一组表示科学计量和具有统计意义数字中的一位数可以用汉字,如一个人,六条意见。星期几一律用汉字,如星期六。邻近两个数字并列连用,表示概数,应该用汉字数字,数字间不用顿号隔开,如三五天,七八十种,四十五六岁,一千七八百元等。
(4)数字作为词素构成定型的词、词组、惯用语、缩略语等应当使用汉字。如二倍体,三叶虫,第三世界,“七五”规划,相差十万八千里等。
(5)5位以上的数字,尾数零多的,可改写为以万、亿为单位的数。一般情况下不得以十、百、千、十万、百万、千万、十亿、百亿、千亿作为单位。如~\num{345000000}~公里可改写为3.45亿公里或~\num{34500}~万公里,但不能写为3亿~\num{4500}~万公里或3亿4千5百万公里。
(6)数字的书写不必每格一个数码,一般每两数码占一格,数字间分节不用分位号“,”,凡4位或4位以上的数都从个位起每3位数空半个数码(1/4汉字)。“\num{3000000}”,不要写成“3,000,000”,小数点后的数从小数点起向右按每三位一组分节。一个用阿拉伯数字书写的多位数不能从数字中间转行。
(7)数量的增加或减少要注意下列用词的概念:1)增加为(或增加到)过去的二倍,即过去为一,现在为二;2)增加(或增加了)二倍,即过去为一,现在为三;3)超额80\%,即定额为100,现在为180;4)降低到80\%,即过去为100,现在为80;5)降低(或降低了)80\%,即原来为100,现在为20;6)为原数的1/4,即原数为4,现在为1,或原数为1,现在为0.25。
应特别注意在表达数字减小时,不宜用倍数,而应采用分数。如减少为原来的1/2,1/3等。

\section{索引示例}[Index]

为便于检索文中内容,可编制索引置于论文之后(根据需要决定是否设置)。索引以论文中
的专业词语为检索线索,指出其相关内容的所在页码。索引用中、英两种文字书写,中文在
前。\sindex[china]{qi!乔峰}\sindex[english]{Xu Zhu}\sindex[english]{Qiao Feng}
中文按各词汉语拼音第一个字母排序,英文按该词第一个英文字母排序。

\section{术语排版举例}[Glossaries and index]

术语的定义和使用可以结合索引,灵活使用。例如,\gtssbp 是一种应用于狄利克雷过程抽样的算法。下次出现将是另一种格式:\gtssbp 。还可以切换单复数例如:\gscna ,下次出现为:\gscnas 。此处体现了\LaTeX\ 格式内容分离的优势。

\section{定理和定义等}[Theorem]

\begin{theorem}[\cite{ren2010}]
宇宙大爆炸是一种爆炸。
\end{theorem}
\begin{definition}[(霍金)]
宇宙大爆炸是一种爆炸。
\end{definition}
\begin{assumption}
宇宙大爆炸是一种爆炸。
\end{assumption}
\begin{lemma}
宇宙大爆炸是一种爆炸。
\end{lemma}
\begin{corollary}
宇宙大爆炸是一种爆炸。
\end{corollary}
\begin{exercise}
宇宙大爆炸是一种爆炸。
\end{exercise}
\begin{problem}[(Albert Einstein)]
宇宙大爆炸是一种爆炸。
\end{problem}
\begin{remark}
宇宙大爆炸是一种爆炸。
\end{remark}
\begin{axiom}[(爱因斯坦)]
宇宙大爆炸是一种爆炸。
\end{axiom}
\begin{conjecture}
宇宙大爆炸是一种爆炸。
\end{conjecture}

\lipsum[1]

\section{其他杂项}[Miscellaneous]

\subsection{右翻页}[Open right]

对于双面打印的论文,强制使每章的标题页出现右手边为右翻页。规范中没有明确规定是否是右翻页打印。模板给出了右翻页选项。为了应对用户的个人喜好,在希望设置成右翻页的位置之前添加\cs{cleardoublepage}命令即可。

\subsection{算法}[Algorithms]

算法不在规范中要求,在hitszthesis.sty中有相关定义,一个例子如算法\ref{alg:rerank}所示。
\begin{algorithm}
  \DontPrintSemicolon
  \wuhao
  \caption{混合重排算法}
  \label{alg:rerank}
  \KwData{$A$:待重排的元素集合 \newline
  $\alpha$: 对多样性,相关性作折中的权重因子}
  \KwResult{$A_k$: a subset of $A$ of size k}
\end{algorithm}

\subsection{脚注}[Footnotes]

不在再规范\footnote{规范是指\PGR\ 和\UGR}中要求,模板默认使用清华大学的格式。

\subsection{源码}[Source code]

也不在再规范中要求。如果有需要最好使用minted包,但在编译的时候需要添加“-shell-escape”选项且安装pygmentize软件,这些不在模板中默认载入,如果需要自行载入。

\subsection{思源宋体}[Siyuan font]
如果要使用思源字体,需要思源字体的定义文件,此文件请到模板的开发版网址github:
\href{https://gihitb.com/YangLaTeX/hitszthesis}{https://gihitb.com/YangLaTeX/hitszthesis}
处下载。

\subsection{术语词汇管理}[Manage glossaries]

推荐使用glossaries包管理术语、缩略语,可以自动生成首次全写,非首次缩写。

\section{本章小结}[Brief summary]

\lipsum[2]
