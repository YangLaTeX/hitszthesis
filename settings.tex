% settings.tex
% 储存配置信息

% 修改表格字号,注意tabular要放在table环境里
\BeforeBeginEnvironment{tabular}{\wuhao}
% 修改表格与后文间距,缩短1cm
\AfterEndEnvironment{table}{\vspace{-1cm}}

% 新计数器,编排表格编号用
\newcounter{rowno}

% 使公式图表的编号随章节编号更新
\numberwithin{equation}{chapter}
\numberwithin{figure}{chapter}
\numberwithin{table}{chapter}

% 设置图表公式编号格式为1-1
\renewcommand{\thefigure}{\arabic{chapter}-\arabic{figure}}
\renewcommand{\thetable}{\arabic{chapter}-\arabic{table}}
\renewcommand{\theequation}{\arabic{chapter}-\arabic{equation}}

% 使用tikz配置流程图基本图形
\usetikzlibrary{shapes.geometric, arrows}
% 开始
\tikzstyle{startstop} = [rectangle, rounded corners, minimum width = 2cm, minimum height=1cm,text centered, draw = black]
% 输入输出
\tikzstyle{io} = [trapezium, trapezium left angle=70, trapezium right angle=110, minimum width=2cm, minimum height=1cm, text centered, draw=black]
% 过程
\tikzstyle{process} = [rectangle, minimum width=3cm, minimum height=1cm, text centered, draw=black]
% 判断
\tikzstyle{decision} = [diamond, aspect = 3, text centered, draw=black]
% 箭头形式
\tikzstyle{arrow} = [->,>=stealth]

% 参考文献标号为上标
\newcommand{\upcite}[1]{\textsuperscript{\textsuperscript{\cite{#1}}}}

% 设置脚注编号格式
\renewcommand{\thefootnote}{\fnsymbol{footnote}}

%%=====新数学命令================================
\newcommand\dif{\text{d}}
\newcommand\no{\noindent}
\newcommand\dis{\displaystyle}
\newcommand\ls{\leqslant}
\newcommand\gs{\geqslant}

\newcommand\limit{\dis\lim\limits}
\newcommand\limn{\dis\lim\limits_{n\to\infty}}
\newcommand\limxz{\dis\lim\limits_{x\to0}}
\newcommand\limxi{\dis\lim\limits_{x\to\infty}}
\newcommand\limxpi{\dis\lim\limits_{x\to+\infty}}
\newcommand\limxni{\dis\lim\limits_{x\to-\infty}}
\newcommand\limtpi{\dis\lim\limits_{t\to+\infty}}
\newcommand\limtni{\dis\lim\limits_{t\to-\infty}}

\newcommand\sumn{\dis\sum\limits_{n=1}^{\infty}}
\newcommand\sumnz{\dis\sum\limits_{n=0}^{\infty}}

\newcommand\sumi{\dis\sum\limits_{i=1}^{\infty}}
\newcommand\sumiz{\dis\sum\limits_{i=0}^{\infty}}
\newcommand\sumin{\dis\sum\limits_{i=1}^{n}}
\newcommand\sumizn{\dis\sum\limits_{i=0}^{n}}

\newcommand\sumk{\dis\sum\limits_{k=1}^{\infty}}
\newcommand\sumkz{\dis\sum\limits_{k=0}^{\infty}}
\newcommand\sumkn{\dis\sum\limits_{k=0}^n}
\newcommand\sumkfn{\dis\sum\limits_{k=1}^n}

\newcommand\pzx{\dis\frac{\partial z}{\partial x}}
\newcommand\pzy{\dis\frac{\partial z}{\partial y}}

\newcommand\pfx{\dis\frac{\partial f}{\partial x}}
\newcommand\pfy{\dis\frac{\partial f}{\partial y}}

\newcommand\pzxx{\dis\frac{\partial^2 z}{\partial x^2}}
\newcommand\pzxy{\dis\frac{\partial^2 z}{\partial x\partial y}}
\newcommand\pzyx{\dis\frac{\partial^2 z}{\partial y\partial x}}
\newcommand\pzyy{\dis\frac{\partial^2 z}{\partial y^2}}

\newcommand\pfxx{\dis\frac{\partial^2 f}{\partial x^2}}
\newcommand\pfxy{\dis\frac{\partial^2 f}{\partial x\partial y}}
\newcommand\pfyx{\dis\frac{\partial^2 f}{\partial y\partial x}}
\newcommand\pfyy{\dis\frac{\partial^2 f}{\partial y^2}}

\newcommand\intzi{\dis\int_{0}^{+\infty}}
\newcommand\intd{\dis\int}
\newcommand\intab{\dis\int_a^b}

\newcommand{\degree}{^\circ}

\newcommand\ma{\mathcal{A}}
\newcommand\mb{\mathcal{B}}
\newcommand\mc{\mathcal{C}}
\newcommand\me{\mathcal{E}}
\newcommand\mg{\mathcal{g}}

\newcommand\mcc{\mathbb{C}}
\newcommand\mrr{\mathbb{R}}
\newcommand\mzz{\mathbb{Z}}

\newcommand\mx{\bf{x}}
\newcommand\mX{\bf{X}}
\newcommand\my{\bf{y}}
\newcommand\mY{\bf{Y}}
%%=============================================

%%=====定义新数学符号===============================
\DeclareMathOperator{\sgn}{sgn}
\DeclareMathOperator{\arccot}{arccot}
\DeclareMathOperator{\arccosh}{arccosh}
\DeclareMathOperator{\arcsinh}{arcsinh}
\DeclareMathOperator{\arctanh}{arctanh}
\DeclareMathOperator{\arccoth}{arccoth}
\DeclareMathOperator{\grad}{\bf{grad}}
\DeclareMathOperator{\argmax}{argmax}
\DeclareMathOperator{\argmin}{argmin}
\DeclareMathOperator{\diag}{diag}
\DeclareMathOperator{\csign}{csign}
%===============================================
